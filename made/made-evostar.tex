\documentclass[runningheads,a4paper]{llncs}

\usepackage[latin1]{inputenc}
\usepackage{amssymb}
\usepackage{amsmath}
\setcounter{tocdepth}{3}
\usepackage{graphicx}
\usepackage{multirow}
\usepackage{rotating}
\usepackage{subfigure}
%\usepackage{subfig}
\usepackage{url}
\usepackage{caption}

\newcommand{\keywords}[1]{\par\addvspace\baselineskip
\noindent\keywordname\enspace\ignorespaces#1}

\providecommand{\tabularnewline}{\\}

\begin{document}

\mainmatter
\title{Title under construction}


\titlerunning{}

%\author{R. H. P. Garc\'ia-Ortega \and P. Garc\'ia-S\'anchez \and J.J. Merelo}
\author{Mustrum Ridcully\inst{1}}
% �Se puede ayudar aqu�?
%
%\authorrunning{R. H. P. Garc\'ia-Ortega et al.}
\authorrunning{}
% (feature abused for this document to repeat the title also on left hand pages)

% the affiliations are given next; don't give your e-mail address
% unless you accept that it will be published
%\institute{Dept. of Computer Architecture and Technology, University
%of Granada, Spain}
\institute{Unseen University of Ankh-Morpork}

%
% NB: a more complex sample for affiliations and the mapping to the
% corresponding authors can be found in the file "llncs.dem"
% (search for the string "\mainmatter" where a contribution starts).
% "llncs.dem" accompanies the document class "llncs.cls".
%




\maketitle

%
%%%%%%%%%%%%%%%%%%%%%%%%%%%%%%%   ABSTRACT   %%%%%%%%%%%%%%%%%%%%%%%%%%%%%%%
%
\begin{abstract}
The procedural generation of massive subplots and backstories in secondary characters that inhabit Open World videogames usually lead to stereotyped characters that act as mere decoration of the virtual world; however, many game designers claim that the stories can be very relevant for the player?s experience. For this reason we are looking for a methodology that improves the variability of the characters? personality while enhancing the quality of their backstories following artistic or literary guidelines.
In previous works, we used multi agent systems in order to have stochastic but regulated inter-relations that became backstories; later, we have used genetic algorithms to promote the appearance of high level behaviors inside them.
Our current work follows the previous research line and propose a three layered system (Evolutionary computation - Agent-Based Model - Logical Reasoner) that is applied to the promotion of the Monomyth, commonly known as the hero?s journey, a social pattern that constantly appears in literature, films, and videogames. 
As far as we know, there is no previous attempt to model the monomyth as a logical theory, and no attempt to use the sub-solutions for narrating purposes. Moreover, this paper shows for the first time this multi-paradigm three-layered methodology to generate massive backstories. 
Different metrics have been tested in the experimental phase, from those that sum all the monomyth-related tropes to those that promote distribution of archetypes in the characters.

% Results are coming...

\keywords{Procedural generation, monomyth, agent-based model, emergent behavior}
\end{abstract}

%
%%%%%%%%%%%%%%%%%%%%%%%%%%%%%%%   INTRODUCTION   %%%%%%%%%%%%%%%%%%%%%%%%%%%%%%%
%
\section{Introduction}

\maketitle

%%%%%%%%%%%%%%%%%%%%%%%%%%%%%%  STATE OF THE ART  %%%%%%%%%%%%%%%%%%%%%%%%%%%%%
%
\section{State of the Art}
\label{sec:SoA}



%
%%%%%%%%%%%%%%%%%%%%%%%%%%%%%%  METHODOLOGY  %%%%%%%%%%%%%%%%%%%%%%%%%%%%%
%
\section{Methodology and experimental setup}


% --------------------------------------------------------------


%%%%%%%%%%%%%%%%%%%%%%%%%%  EXPERIMENTS AND RESULTS  %%%%%%%%%%%%%%%%%%%%%%%%%%
%
\section{Experiments and Results}

\section{Conclusions}


\section*{Acknowledgments}
Hidden for double blind review 
%\scriptsize{This work has been supported in part by SIPESCA (Programa Operativo FEDER de Andaluc\'ia 2007-2013), TIN2011-28627-C04-02 (Spanish Ministry of Economy and Competitivity), SPIP2014-01437 (Direcci\'on General de Tr\'afico), PRY142/14 (Fundaci\'on P\'ublica Andaluza Centro de Estudios Andaluces en la IX Convocatoria de Proyectos de Investigaci\'on) and PYR-2014-17 GENIL project (CEI-BIOTIC Granada).}


%
%Hidden for double-blind review


\bibliographystyle{splncs}
\bibliography{hpmoon-evostar}


\end{document}
